%Tex-Stuff
\documentclass[a4paper,12pt]{article}

%Imports
\usepackage[pdftex]{graphicx}
\usepackage[utf8]{inputenc}
\usepackage{subcaption}
\usepackage{hyperref}


%%%%%%%%%%%%%%%%%%%%%%%%%%%%%%%%%%%%%%%%%%%%%%%%
\begin{document}

\section{PaperOverflow: Produktivision}
Die Beschreibung der Produktvision erfolgt noch den Feldern des  \href{http://www.romanpichler.com/tools/vision-board/}{"Product-Vision-Boards"} von Roman Pichler. 
\begin{figure}[h!]
  \centering
  \includegraphics[width=0.7\linewidth]{res/PVB.pdf}
\end{figure}

Besonderes Augenmerk soll dabei auf die Produktbeschreibung gelegt werden.
Die Geschäfts-Ziele hingegen, sollen aufgrund der geringen Relevanz für das Projekt nicht beleuchtet werden.

\subsection{Vision}
Ziel von PaperOverflow ist es den Autoren von wissenschaftlichen Arbeiten im Bereich der Literaturverwaltung zu unterstützen. Dabei soll die PaperOverflow im Gegensatz zu ähnlichen Produkten komfortabel und unkompliziert sein, fehlerhafte Zitate vermeiden, ein Aufsuchen von zitierbaren Textstellen ermöglichen und mit wachsender Menge an zu verwaltenden Dokumenten besser werden.

\subsection{Zielgruppe}
Wer sind die Anwender von PaperOverflow?\\
Zunächst wird PaperOverflow von Benutzern verwendet, deren Ziel es ist eine wissenschaftliche Arbeit zu verfassen. PaperOverflow soll für diese Benutzer die Verwaltung von wissenschaftlichen Veröffentlichungen anderer Autoren übernehmen.\\
Vorrangig soll das Auditorium von PaperOverflow aus Anwendern bestehen, welche zum Verfassen der Wissenschaftlichen Arbeit LaTeX verwenden. Jedoch sollen auch Nicht-LaTeX-Benutzer PaperOverflow zur Verwaltung von Veröffentlichungen nutzen können.\\
PaperOverflow soll zum einen von Benutzern verwendet werden, die technisch versiert sind und gerne mit komplexer wirkenden, dafür jedoch mächtigeren Oberflächen arbeiten. Zum anderen sollen auch weniger technisch versierte Nutzer durch eine einfache Oberfläche angenehm mit PaperOverflow arbeiten können.\\
Des Weiteren werden die Anwender von PaperOverflow verscheiden große Mengen an Veröffentlichungen zu verwalten haben. Dabei soll sich der Einsatz der Software schon ab wenigen Dokumenten lohnen, aber auch riesige Mengen an Veröffentlichungen handhaben können. Das Verhältnis, welche Dokumente in digitaler Form vorliegen und welche als haptisches Exemplar vorhanden sind, variiert dabei ebenfalls von Nutzer zu Nutzer.\\
\\
Um die oben genannten variierenden Eigenschaften der Zielgruppe zu veranschaulichen, sollen als Vertreter der gesamten Zielgruppe drei Personas aufgelistet werden.

\paragraph{Wolfgang der Word-User mit wenigen Dokumenten}
- Wolfgang möchte einen kurzen wissenschaftlichen Artikel veröffentlichen und hat sich entscheiden dies in Microsoft Word zu tun. Er benötigt nur eine Hand voll wissenschaftlicher Quellen, die er richtig zitiert. Diese Quellen liegen ihm teils digital, teils analog vor. Er ist nicht weiter technisch versiert und es ist auch nicht zu erwarten, dass er in näherer Zeit weitere wissenschaftliche Arbeiten schreiben wird.

\paragraph{Larissa die LaTeX-Userin mit vorrangig digitalen Dokumenten}
- Larissa nutzt LaTeX, um ihre Abschlussarbeit anzufertigen. Sie setzt sich dafür zum ersten Mal intensiver mit LaTeX und BibTeX auseinander. Sie hat für Ihre Arbeit etwa 40 potentielle Quellen, von denen Sie voraussichtlich etwa 30 zitieren wird. Die Quellen liegen Larissa hauptsächlich in digitaler Form vor.

\paragraph{Prof. Probst der LaTeX-Power-User mit vorrangig analogen Dokumenten}
- Prof. Probst ist LaTeX-Vollprofi und verfasst regelmäßig wissenschaftliche Dokuemnte zu verschiedenen Themen und verschiedenen Fachrichtungen. Er hat inzwischen eine Dokumentensammlung von 1200 wissenschaftlichen Veröffentlichungen. Davon stehen die meisten in seinem Bücherregal, nur wenige liegen ihm in digitaler Form vor.

\subsection{Bedürfnisse}

Alle drei oben genannten Persona haben generell das Bedürfnis Literatur in digitaler oder analoger Form einfach und unkompliziert als Ressource zu erfassen, in den erfassten Ressourcen komfortabel nach Inhalten suchen zu können und die benötigten Textstellen mit wenigen Klicks korrekt und dem verwendetem Ziteirstandard entsprechend zu ziteren.\\

Dabei steht für Persona Wolfgang vor allem die Einfachheit aller Schritte im Vorderugrund. Für diesen Nutzer müssen vor allem simple Möglichkeiten für das schnelle Zitieren und eine ansprechende, simple Benutzeroberfläche gegeben sein. Die Operationen um Literatur zu erfassen, Zitate zu finden und diese korrekt zu zitieren müssen für Wolfgang auf ein Minimum reduziert werden können.\\

Für Persona Larissa liegt das besondere Augenmerk auf dem einfachen und komfortablen Hinzufügen von wissenschaftlichen Dokumenten im PDF sowie dem komfortablen zitieren von Stellen in diesen Dokumenten. Die gefundenen Zitate sollen in einem Set zusammengefasst werden, welches im BibTeX-Format exportiert werden soll. Larissa benötigt eine größere BibTeX-Datei, welche alle zu zitierenden Stellen enthält. Diese Stellen soll Sie innerhalb des LaTeX-Dokuments referenzieren können.\\

Persona Prof. Probst benötigt vor allem eine Möglichkeit seine 1000 analog vorliegenden wissenschaftlichen Dokumente schnell und einfach im PaperOverflow zu erfassen. Besonders ist dabei eine Unterstützung beim Hinzufügen von Stichworten zu den Dokuemnten benötigt, um später möglichst genaue Treffer bei der Suche zu erhalten. Beim Erfassen ist zusätzlich eine Unterteilung der erfassten Dokumente in Kategorien notwendig, um später die Suche genauer eingrenzen zu können. Beispielsweise ist eine solche Unterteilung nach Themengebiet benötiogt, wenn Prof. Probst in zwei verscheidenen Fachbereichen arbeitet und die Dokumente des einen Fachbreriches keine Relevanz für den anderen haben. Eine Zuordnung zu Kategorien möchte der Prof. sowohl beim Import als auch im Nachgang vornehmen können. \\
Beim Suchen von Dokumenten benötigt Prof. Probst sowohl eine Kategorie-spezifische Suche, als auch eine Suche in der kompletten Menge der erfassten Dokumente. Für Prof. Probst als technisch versierten Benutzer mit vielen Dokumenten soll eine erweiterte Suche möglich sein, durch die sich Suchanfragen genauer spezifizieren lassen.\\
Ist ein Dokument gefunden, benötigt Prof. Probst Empfehlungen, welche weiteren Dokuemnte, die sich im PaperOverflow befinden, für Ihn von Interesse sein können.\\
Für das Zitieren benötigt Prof. Probst analog zu Persona Larissa eine Möglichkeit Zitate zu einem Set zusammenzufassen und später als BibTeX-Datei zu exportieren. Dabie benötigt Prof. Probst mehrere verschiedene Sets an Zitaten, für den Fall dass er an mehreren Arbeiten arbeitet. Die Kategorisierung der Sets solll dabei durch Prof. Probst konfigurabel sein.

\subsection{Produktbeschreibung}

Sie haben sich dazu entschieden eine wissenschaftliche Arbeit zu verfassen und sind nun dabei sich Wissen zum Themengebiet heranzuziehen? Und der Haufen der Veröffentlichungen wir immer größer und größer? Und ganz allmählich unbezwingbar? Sie nähern sich langsam der Grenze, ab der sie einfach nicht mehr Dokumente handhaben können? Dem Overflow an Papers? 
\begin{center}
Keine Panik! 
\end{center}
Der Overflow ist nicht böse, er braucht nur die richtige Handhabe. Mit PaperOverflow holen Sie das Maximum aus Ihrer Dokumentensammlung heraus!\\ \\
PaperOverflow unterstützt Sie in Ihrem Arbeiten, in den drei folgenden Schritten:

\renewcommand\thesubfigure{\arabic{subfigure}}
\begin{figure}[h!]
  \centering
  \begin{subfigure}[b]{0.32\linewidth}
    \includegraphics[width=\linewidth]{res/step1.png}
    \caption{Den PaperOverflow nähren.}
    %\label{fig:1} 
  \end{subfigure}
  \begin{subfigure}[b]{0.32\linewidth}
    \includegraphics[width=\linewidth]{res/step2.png}
    \caption{Den PaperOverflow durchsuchen}
    %\label{fig:2} 
  \end{subfigure}
  \begin{subfigure}[b]{0.32\linewidth}
    \includegraphics[width=\linewidth]{res/step3.png}
    \caption{Den PaperOverflow zitieren}
    %\label{fig:3} 
  \end{subfigure}
\end{figure}



\paragraph{Den PaperOverflow nähren.}\

PaperOverflow unterstützt Sie bei dem Erfassen Ihrer Dokumente. \\

\textbf{Dokumente in gedruckter Form:} 
\begin{itemize}
	\item Holen Sie das Dokument aus dem Schrank
	\item Erfassen Sie über die ISBN-Nummer, den Titel oder den Autor mit Unterstützung einer Auto-Vervollständigung Ihr Dokument als Paper.
	\item PaperOverflow bietet Ihnen Vorschläge an, um welches Dokument es sich handelt. Wählen Sie eine Alternative und bestätigen Sie. 
	\item Eine manuelle Korrektur oder Vervollständigung der Daten ist jederzeit möglich.
	\item PaperOverflow schlägt Ihnen Stichworte für den Dokumenteninhalt an. Diese sind für die spätere Suche nach Dokumenten nützlich. Wählen sie passende Stichworte aus oder fügen Sie komfortabel weitere hinzu.
	\item Mit wenigen Klicks ist damit Ihr Dokument erfasst.
\end{itemize} \

\textbf{Dokument als Datei:}
\begin{itemize}
	\item Legen Sie das Dokument per Drag and Drop im PaperOverflow ab.
	\item Sobald Sie das Dokument abgelegt haben holt PaperOverflow für Sie die Meta-Informationen und bietet Ihnen einen Vorschlag an, um welches Dokument es sich handelt. Bestätigen Sie oder wählen Sie ggf. eine Alternative.
	\item Eine manuelle Korrektur oder Vervollständigung der Daten ist jederzeit möglich.
	\item PaperOverflow sucht Ihnen die Stichworte für den Dokumenteninhalt aus dem Dokument und bietet diese Als Vorschlag an. Wählen sie passende Stichworte aus oder fügen Sie komfortabel weitere hinzu.
	\item Ist das Dokument im PaperOverflow abgelegt, Können Sie es bei wieder als Datei exportieren.
\end{itemize}\ 

\textbf{Ihr PaperOverflow wird groß? $\to$ Nutzen Sie PaperStacks!}
\begin{itemize}
	\item Ordnung von Beginn an: Legen Sie für Ihre Dokumente verscheidene PaperStacks an! Damit lassen sich Dokumente für gänzlich verschiedene Themenbereich von Beginn an voneinander abtrennen.
	\item Schon beim Zufügen von Dokumenten zum PaperOverflow ist das Zuordnen von Papers da optional wählbar
	\item Nachher auch sortierbar
\end{itemize}

Alles zusammen schön übersichtlich.
Sie sind technisch versiert, lesen BibteX fließend und möchten mehr sehen? Schalten Sie um auf den wiss mod!
\\
\paragraph{Den PaperOverflow durchsuchen}\

Dann sitzen Sie da und suchen nach guten Stellen.
\begin{itemize}
	\item Auto-Vervollständigung bei Suche
	\item Komfortabel Suchen in Stichworten, Abstract, Titeln, Autoren. (iTunes)
	\item erhalten Sie Vorschläge für ähnliche Dokumente
	\item historisierte Suchanfragen
\end{itemize}

Das Dokument liegt digital vor? Zeigen Sie es an und lesen Sie rein!

\paragraph{Den PaperOverflow zitieren}\ 

Passendes Zitat in Ihrem Overflow gefunden? \\

\textbf{Zitieren Sie!} 
\begin{itemize}
	\item In Dokument gleich mit Zeilenangabe
	\item aber auch für print-Exemplare schnell angegeben.
\end{itemize}

\textbf{Set bauen}
\begin{itemize}
	\item QuickQuote: Mit einem Klick haben Sie den entsprechenden BibTex in der Zwischenablage und können Ihn weiterverarbeiten.
	\item Oder: Zu Zitaten hizufügen und weitersuchen.
	\item Dabei: Info hizufügen.
	\item Sie haben mehr als eine Arbeit, für die Sie zitate sammeln? Legen Sie Zitat-Sets an!
\end{itemize}

\textbf{Export!}
\begin{itemize}
	\item Sets rausgeben in gängigen Exportformaten: ...
	\item Export als Referenzlisten: ..
	\item Zusammenarbeit mit anderen? Per Mail senden: Zitat, Refliste.
\end{itemize}



\pagebreak
\section{PaperOverflow: Vokabeln}
\begin{tabular}{ p{5cm} p{8.5cm} }
  \textbf{Paper} & Das zum PaperOverflow hinzugefügte wissenschaftliche Dokument. Dieses enthält entweder nur die Metadaten oder die Metadaten und zusätzlich die PDF-Datei. \\ \\
  \textbf{PaperOverflow} & Zum einen das Programm selbst. Zum anderen bezeichnet der PaperOverflow den alle Papers im Programm (alle im Programm erfassten Dokumente) \\ \\
  \textbf{PaperStack} & Bezeichnet einen Teil des PaperOverflows. Also nur einen kleinen Stapel an Papers. Verscheidene PaperStack können genutzt werden, um Dokumente verschiedener Kategorien voneinander abzutrennen.\\ \\
  \textbf{Quote} & Eine zitierte Stelle wird als Quote gespeichert.\\ \\
  \textbf{QuoteSet} & Mehrere Quotes lassen sich zu einem Set an Quotes, einem QuoteSet, zusammenfassen.\\
\end{tabular}



\end{document}
%%%%%%%%%%%%%%%%%%%%%%%%%%%%%%%%%%%%%%%%%%%%%%%%