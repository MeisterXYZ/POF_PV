%Tex-Stuff
\documentclass[a4paper,12pt]{article}

%Imports
\usepackage[pdftex]{graphicx}
\usepackage[utf8]{inputenc}
\usepackage{subcaption}
\usepackage{hyperref}


\begin{document}

\section{In Produktivision nicht (näher) expliziert:}

\subsection{Aus Aufgabenstellung:}

\paragraph{Die Software muss Meta-Daten von wissenschaftlicher Literatur verwalten}
\begin{itemize}
\item Typen:Buch,Tagungsband,Tagungsbeitrag,Zeitschriftenbeitrag,TechnischerBericht,Mas- terarbeit, Doktorarbeit, Sonstiges
\item alle relevanten Felder (vgl Bibtex) sind zu verwalten 
\item Abstract und Schlüsselworte sind zu erfassen 
\end{itemize} 
\paragraph{Zu einem Dokumente sollen alle anderen Dokumente der (oder eines der) Autoren anzeigbar sein}
\paragraph{Bibtex-Einträge müssen einer Plausibilitäts- und Konsistenzprüfung unterzogen werden}
\paragraph{Zu Upload von PDF-Papern:}
\begin{itemize}
\item Bezüglich der Schlüsselworte sollen alle nicht-trivialen Worte des Systems erhoben werden, per Unschärfe zB Singular und Plural oder einfache Konjugationen bei Verben aufeinander abgebildet und die 20 wichtigsten Begriffe des Texts erhoben werden
\item Ebenfalls sinnvolle wäre eine automatisierte Erkennung von Literaturreferenzen Hier ist das Ziel Verweiseauf andere in der Software abgelegte Dokumente zu erkennen
\item die weiteren bibliographischen Informationen mussen von Hand gepflegt werden
\end{itemize}
\paragraph{Die Darstellung der Schlüsselworte als Tag-Cloud zu einem Dokument ist sinnvoll}

\subsection{Aus Workshop:}
\paragraph{Web oder Desktop?}
Erste Zuarbeit siehe Aufgabe 1
\paragraph{Wechsel zwischen einfachem und wissenschaftlichem Modus}
\paragraph{Unterstützung aller Betriebssysteme}
\paragraph{Datenbankverbindungen zu Standarta}
\begin{itemize}
\item PubMed
\item vielleicht ArXiv
\end{itemize}

\paragraph{Netzwerkfunktionalität}

\paragraph{nice-to-have}
\begin{itemize}
\item OCR Funktionalitäten
\item Handyapp verbindet sich mit Netzwerkschnittstelle und wird zum Scanner für
Buchtitel, ISBN Nummern, Titeln, Covern... $\to$ diese können eingepflegt werden
\end{itemize}

\paragraph{Nutzung von Systemviewern}

\paragraph{Suchfeld muss intelligent sein}
\begin{itemize}
\item lernt dazu
\item Schlägt Kategorie in der gesucht werden soll vor
\item Auto-vervollständigung
\end{itemize}


\end{document}